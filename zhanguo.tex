\documentclass[UTF8]{ctexart}
\usepackage{ulem}
\begin{document}
自汉家天下三分,纷扰征逐终归于晋而后安,已四十年矣,当此之时,天子坐东都,诸侯镇九州,百姓久享太平,不知战苦。其时晋家天下,绵延数千里,北起东辽,南至交趾,西接吐蕃,东望东海,四海之内皆王土也。晋天子遵祖训裂土而封王,有十二亲王,为秦,汉,赵,肃,燕,韩,蜀,齐,魏,吴,唐,鲁,又有十二郡王,为九江,济南,信北,怀义,靖南,平德,新平,平西,绥德,襄阳,博远。大者数州,小者一郡,非皇室功戚不王之。


秦王势大,不臣久来,\uline{上}常欲废之。咸德四年春,\uline{上}宴华盖殿,珍馐美馔,红裙绿舞,召文武诸侯列席。汉王,肃王,魏王年迈,不能从封地远至,皆遣王使致意。齐王,蜀王,唐王,鲁王藩国路远,不能远至,亦皆遣王使致意。韩王,吴王,燕王接诏之日,即动身进京,奉诏领半数护卫营入觐。唯有秦王,赵王,不奉诏,不遣使,只各上了一封表,大意重责在身,不可轻易离藩。\uline{上}席间览表不悦,问左右太监“秦赵倨傲,谁可治之”,太监唯唯不敢言,\uline{上}命传表群臣,众臣俯伏。只见文班之中有一青年,面如皓月,眉如立剑,身着三品麒麟官服,举白牙象笏出列,奏曰“太宗祖制,诸侯就藩而屏蔽皇室,如众星散布而能拱月,为防懈怠,无敕令不可离境,今秦赵表奏,不尽无理,陛下察之”。\uline{上}视之,乃少卿萧琦。又有武班之中一老者,须发皆白,身形健朗,身着二品猩红大袍,亦举白牙象笏出列,奏曰“陛下已诏令入京觐见,何谓 ’无敕令’,二王见诏犹自怠慢,无礼至极,宜论罪”。乃剑阁将军孟爽。\uline{上}令舞姬退散,扶额坐于金座之上,久之乃罢宴。当日群臣各自回邸,\uline{上}独留孟爽,屏退左右,对坐于殿上,孟爽惶恐,数起身,\uline{上}乃执其手,叹曰“今日大宴之上,唯卿之言,能慰朕意,秦王我叔,纵横十五郡,雍州土地十之五六,凉州土地十有七八皆置其掌中,更兼麾下被坚执锐之士十余万,已成朝廷心腹之患,一旦有变,关中朝夕不保矣。”\uline{爽}闻言拱手曰“秦虽强,不过一方之霸,朝廷兵精将广,粮银充足,若秦不轨,臣愿为先锋,不破咸阳终不还”,\uline{上}意稍解,随命左右取一金箧,放于殿中,谓\uline{爽}“我十年之前亲选数名干练精明之能人,放于秦府之中,数年以来为我心腹,亦为秦王倚重,盖能为我督查秦主一举一动。今朕决意削藩,且从秦始,若诸王服从,卿当颐享天年,若不然,此盒中诸人,与卿一道,当勉力为朕忧!” 孟爽闻言不觉一惊,“ \uline{上}以密夾授我,我不能开,开则从此祸害无穷,再难逃干系” 一时不知如何回话,再转眼看这华盖殿,白日鼓乐隆隆,人来人往,故不觉殿内空旷,现独君臣二人与殿角两三内侍,烛影飘摇,顿觉幽戚。良久,\uline{爽}只顿首伏地不起。\uline{上}怪之 “卿不愿为朕谋耶?” \uline{爽}乃接箧奏曰“陛下放手削藩,臣敢不肝脑涂地以助陛下!” \uline{上}悦,扶\uline{爽}起,令人赐九章缎带,明日下旨,除\uline{爽}为镇西将军,封明意侯。


于是京中显要,皆知\uline{上}不满秦赵二王。


是年夏五月,秦世子在东都作奸不法,掳人妻女,\uline{上}大怒,命有司论罪,收押下狱,更遣使至咸阳责问秦王过,言辞激烈。夏六月,秦王上表为世子乞怜,\uline{上}不许,徐命废秦世子为庶人。秦王再表谢罪,\uline{上}又下诏责。夏七月,\uline{上}令除咸阳天策卫、地佑卫各两百护卫营,拨与雍州督护府,命秦王自省。


于是天下显要,皆知\uline{上}欲除秦而后快。

秦王召幕僚谋臣,议于咸阳府,“孤先帝长兄,当今皇叔,自十七岁就藩以来,无有狂悖僭行,一心为国镇守西陲。今上登基十有四年,无一日不以逆臣之心度我,孤履上表自清,\uline{上}皆不采。今岁华盖之宴,\uline{上}为歌舞之乐竟召天下诸侯离国远赴,实非人君之行,我以先帝祖宗法令,无特殊不得擅离藩国,实非藐视陛下,奈何从此不可收拾,先废我世子,后除我护卫,祸不远矣。”上卿杜远奏曰“大王少忧,秦地广物丰,民富兵壮,为国雄镇西北,为诸王第一,陛下虽有微怨,不至宗亲反目,君臣列阵。”话音未落,王座之后有女音娇斥道“杜道远之言差矣!古云叶落知秋,管中可以窥豹,今上不顾宗室亲情之意已明,我兄有罪,罪不至庶人,\uline{上}但念一丝兄弟亲亲之意,何至于此。”众人惊讶,但见秦王身后一紫衣少女,不施粉黛,只披薄纱素裙,踱步出殿,一双清婉柔目自射一股怒气,直视杜远,原乃秦国长公主尚芝。众臣皆知此女少不学女红,长不读女学,素日在宫中飞扬跋扈,不服教养,秦王又多宠溺,偌大王城竟无一人可以管的她去,此时闻言皆不作声。 秦王啐道“军国大势,女子何知?休得乱语” 尚芝不服,把怀中匕首掷于王座之前,抱手道“朝廷削藩之难,如此匕首,迅捷如斯,已在阿父座前,当早做决断,不为人后” 王默然,次卿张怡出奏曰“天下久安久治,民心恶战,朝廷必不背民意而起干戈,此自绝于百姓”,次卿刘铎亦奏“朝廷总领天下军机,良马何止十万,带甲何止百万,秦诸王之中虽为翘楚,然欲东向与朝廷争,此以卵击石也。大王虽镇西陲,我朝法制,藩王虽统军不治府,雍凉各郡守素只知洛阳,不知咸阳,自立国以来不奉秦命,一旦有事,必竭力为朝廷剿我,大王不可急躁。” 长公主恼怒,箭步至刘铎身边,伸手折了刘铎的象牌,斥道“此祸国祸主之言!是要阿父坐以待毙之言,是要我秦府坐等屠灭之言!各郡军力皆在我手,各郡守如何为朝廷剿我?况阿父为王数十载,各郡皆有心腹,彼名虽晋臣,实为秦卿,到时必唯秦命是从罢了” 刘铎惊吓,匆匆捡起被折成数段之象牌碎片,忙欲拜退,秦王止之,厉声斥尚芝无礼,强命女侍扶\uline{芝}回宫。众臣谋士相视无奈,秦王叹息良久,亦起身回宫,众人遂依次告退。

须知咸阳王城占地甚广,自建藩之日起至前朝恒兴二十六年止,耗三十万民兵农役,数百万两白银,日夜建造,二十年始成。有朝圣门,希德门,正合门,东华门四门,文锦殿,武成殿,集贤殿三殿,奉仪宫,中坤宫,毓秀宫,阳朔宫四主宫并东西十二小殿、十二从宫。外圆内方,依咸阳茂青山而建,在城正中高地,能览咸阳府全貌无遗。凡秦王见臣,均在集贤殿,偶遇\uline{王}不便,则在希德门。这日秦王从集贤殿回,想起先前尚芝无礼,心中不悦,又念及世子遭废、削藩事,愈发烦闷愁苦,乃命左右内侍停驾,转毓秀宫,欲见小女儿友芝。盖秦王有一子二女,长子\uline{检}已遭废为庶人,按晋律囚于高桥皇陵,长女尚芝,次女友芝,三人皆秦国夫人胡氏所出,秦王甚自律,平生不近女色,与胡氏成婚以来未曾纳侧室,故世子公主皆同母兄妹。\uline{王}爱二女甚笃,常谓近侍“孤有二女相伴,无复有求矣”,凡公主所欲,\uline{王}一应皆准,至于公主教导,\uline{王}无苛求,任其心性,故长公主跋扈自矜而少公主温婉明顺。\uline{王}至毓秀宫,少公主闻父亲驾到,急急从宫中出迎,不及拜见,\uline{王}便扶起,挽手入宫中坐定。宫人上茶毕,友芝笑道“阿父今日无事,便来我宫中逗我” \uline{王}闻言大笑“小女刁蛮,却来消遣老父” 少公主低头浅笑,傍秦王脚边,\uline{王}乃取腰间玉环,付与道“今日前朝事多,为父来的匆忙,未曾准备,却把此物给你把玩去吧,别看此物轻巧,此是前日你赵王叔派人千里传递送来与我,是大明府新发的宝物” 友芝接过,放在手里看了几下,却返还给秦王,笑道“我不要这宝物,既是赵王叔千里送来的,我不能夺爱,阿父要送宝贝与我,我却只要一样” 王问“却是何物” 友芝盈盈起身,腻倒在\uline{王}怀中,言道“女儿听闻阿姐今日大闹朝堂,阿父定甚恼怒,今长兄受难,阿姐愤怒情有可原,友芝只要阿父赦免阿姐罪过便是天大宝物了” 说罢只是看着秦王。\uline{王}叹道“你有此心,诚属难得,但你姐今日实在荒唐,在大殿之上折辱廷臣,不罚为父无以报群下。念你求情,暂且罚俸,禁足旬日”  父女二人正相谈间,忽有内侍快步从宫外步入,见秦王即拜倒,奏道“有八百里急递至”。\uline{王}命奉上,只见一金丝木小盒,小巧玲珑,盒边三缕彩线,当中一个秦字,正是秦王府之印信。\uline{王}急启之,见一小小纸卷,用红丝缠绕,\uline{王}大惊,此为京中暗线与王府通信之约定,如有事,用蓝线缠绕黄色纸卷,有急事,用黄线缠绕褐色纸卷,有生死攸关之事,用红线绕白色纸卷,今红线白卷,必是十万火急。\uline{王}不及屏退左右,当即启卷,少公主见父王神情凝重,不敢吱声,示意左右内侍先行退下,自己随侍在侧。眼见秦王面色越发沉重,竟至不发一语,如中风一般,少公主慌忙上前扶住,\uline{王}尤神色呆滞,许久方才侧身将纸卷递与公主,竟失声痛哭道“我儿休矣” 友芝闻听,忙接纸卷来看,但见两行小字,却异常醒目“\uline{上}杀世子,下月废秦”。友芝看了一阵眩晕,扑倒在秦王怀中,二人抱头哭泣不止。门外侍从听得宫内哭声异响,慌不迭抢进宫来护驾,秦王此时哪里顾得有奴仆外人在侧,只顾抱住少公主痛哭。侍从见状纷纷匍匐,颤抖不敢言语。一时毓秀宫中气氛悲切肃穆,这匍匐众人中有一小厮唤作徐闯的,见无人敢动,二主又悲哭不止,自忖不能不报主公夫人,竟偷偷爬至宫门口,飞奔中坤宫报信。 

胡夫人闻信,不及更衣,慌至毓秀宫,见少公主花容憔悴,泪痕斑斑,秦王则被左右搀扶,瘫坐在宫中八宝椅上,乃厉声问徐闯“竟是何事!” \uline{闯}如何知晓,又恐回不知更触其怒,突见公主身旁有掉落一卷白纸,似有字迹,急忙上前拾起,转而低头奉予胡夫人。夫人览卷毕,不语良久,乃命扶秦王回驾奉仪宫,少公主随侍,又命长公主禁足,无诏不得出阳朔宫,又命上卿杜远、冯猛、李立,次卿张怡、刘铎、张录,骠骑将军黎准,白信速至希德门下候诏。徐闯及众侍从领命,各去宣诏,胡夫人乃整理衣冠,驾席德门。

咸德四年八月,秦废世子\uline{检}病卒。

于是胡夫人见众臣于希德门,传纸卷与众人览毕,众皆震悚。夫人起身曰“今上无道!我儿何辜,先废后杀!如今大王哀痛不能视事,国又无后嗣,我一介女流,所知者唯仰赖众卿矣。” 杜远奏曰“秦为亲王之长,今上滥杀宗室,当谕诸王共讨之。” 刘铎伏奏曰“世子已死,不可复生,秦欲报仇,何争朝夕!今若传檄诸王共讨,耗时弥久,朝廷之难在一月之内,诸侯援兵必不及至,此策断不可行也。臣计,不如上表请于宗室诸子中择俊杰年少者立为世子,麻痹其意,再整肃三军,命天水部推至武阴,广德部推至上庸,以示秦力,令其有所忌。而后交通诸王,谋定而后动,大事可成。” 夫人闻言赞许,乃着黎准至天水,领天水部两万精骑前进至武阴,着白信至广德,领广德部两万精兵前进至上庸,均以换防为名,军中禁议废世子之事。秦王病倒,昏睡不省人事,胡夫人乃自上表,言语哀切,求请择宗室少年合适者为继子,用杜远为使者,赍黄金千两至洛阳。

《远》入洛阳,直驱明意侯府。晋朝法制,两京中能设府邸者非公侯显贵不可。东京洛阳府非三公,侯爵,公爵者禁设府邸,西京长安府非三公,九卿,公爵,王爵者禁设府邸。孟爽既封侯爵,本可兴建新府,但其



\end{document}